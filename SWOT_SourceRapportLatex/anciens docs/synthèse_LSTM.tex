\section{Introduction aux LSTM}

\begin{itemize}
    \item présentation neural networks
    \item présentation recurrent neural networks
    \item explosion de gradient 
    \item LSTM \\
\end{itemize}


02 février\\

Modélisation numérique, directe et inverse, des écoulements de fleuves & rivières dont la surface est mesurée par satellite. L'objectif est d’estimer le débit des rivières (m3/s) à l'échelle planétaire à partir des seules observations altimétriques (hauteur et largeur de la surface d'eau).\newline

Bases de données :\newline
-	HYDRoSWOT\\
-	PEPSI\\

Variables d’intérêt pour l’explication de la variabilité du débit des rivières :\\
-	L’aire drainée\\
-	Les aires ajoutées\\
-	La largeur\\
-	Le taux de limon\\
-	Sinuosity & Meandwave\\

Réseaux de neurones :\\
-	Vecteur de poids associé à chaque neurone\\
-	Couche : ensemble de neurones n’ayant pas de connexions entre eux\\
-	Perceptron : un neurone d’une couche cachée est connecté en entrée à chacun des neurones de la couche précédente et en sortie à chaque neurone de la couche suivante.\\
-	Estimation des poids par apprentissage (utilisation d’algorithmes d’optimisation)\\
-	Choix des paramètres : nombre de couches cachées, nombre de neurones par couche cachée, nombre maximum d’itérations, erreur maximale tolérée, terme de régularisation, taux d’apprentissage, taille des ensembles d’observations\\
-	Difficile de se faire une idée de la qualité du modèle\\
-	Fixer le nombre de neurones et d’itérations assez grand, et jouer sur le réglage de la pénalisation\\
-	Librairie e1071\\

Deep learning :\\
-	ConvNet > analyse d’image\\
-	LSTM > dimension temporelle à prendre en compte\\
-	Diabolo > détection d’anomalies\\
-	Empilement de couches de neurones\\

Problème initial\\
Bathymétrie, débit, frictions\\
->	Débit, section mouillée (A0 inconnue)\\

Problème inverse\\
Modèles physiques (équations de mouvement et de la masse) , largeur d’eau, aire drainée, hauteur d’eau (pas A0)\\
->	Débit, section mouillée\\

Réseaux de neurones\\
-	Architecture donne le modèle\\
-	Sortie comparée avec la base de données GRDC (minimisation en modifiant les poids des neurones)\\
-	LSTM permet la corrélation temporelle\\

Deux bases de données : 1 est grande mais avec des données « fausses », 1 est petite mais avec des « vraies » données.\\


https://datasciencetoday.net/index.php/fr/machine-learning/148-reseaux-neuronaux-recurrents-et-lstm#:~:text=Les%20r%C3%A9seaux%20de%20longue%20m%C3%A9moire%20%C3%A0%20court%20terme,ont%20des%20retards%20tr%C3%A8s%20longs%20entre%20les%20deux.