\section{Introduction.}

The prediction of water variation is an important issue. The crucial point is to predict river discharge on a global scale for two main reasons.  In some countries or regions, water is a limited resource, and it often depends on the season. On the contrary, in others regions, water can cause disasters and environmental damage such as floods. Therefore, water forecasting is important to manage water consumption in agricultural or industrial sectors, in people's daily lives, or to evacuate a population in case of an imminent threat. In addition to the environmental issues, rivers are at the center of geopolitical tensions. According to the United Nations, "2 billion people live in countries experiencing high water stress" (UN, 2019 \cite{ONU}). From the water stress results a higher water demanding than the available resources. 

Although the water stress is rising, there are still few data on rivers : the data is poorly distributed geographically and temporally (Matt Fry, 2019 \cite{insitudata}). Moreover, only few countries share these resources. Scientists struggles to estimate the space-time water variation of rivers due to theses reasons (Monnier, 2021 \cite{conf_monnier}). They need to know the exact bathymetry value, the friction coefficient and the potential lateral flux to compute a river flow model (Nils Reidar B. Olsen, 2004 \cite{courshydro}). But these data are very often unknown because of the complexity to measure them.

NASA and CNES created Surface Water Ocean Topography (SWOT) mission (NASA, \cite{NASA}). In April 2022, the first satellite studying only water surfaces (rivers and oceans) will be launched. It will collect altimetric data on rivers wider than 50 meters. The SWOT instrument will observe rivers surface topography and measure how water bodies change over time. Through the observations, it will measure the elevation and the width of rivers. The data collection will last for 3 years with 21 days repeat cycle and will cover the majority of the earth.

Given the SWOT data, hydrologists will face an inverse and ill-posed problem. Using altimetric data, they cannot find exactly the river flow. They need an additional data such as the incoming river flow or the unobserved cross-sectional flow area. And the satellite will not be able to capture this data.\newline

However, scientists manage to predict well water variations, but it exists a potential bias. This bias is due to ungauged rivers where we do not have additional data. We aim to estimate river flow using neural networks without knowing the bathymetry. The objective is to have an estimation without bias.\newline

In this paper, we use two different types of neural networks algorithms. The first one is an Artificial Neural Network (ANN), and the second one a Long Short-Term Memory (LSTM) network, which has feedback connections. This feature is the major point which permits us to reduce the bias.

First, a brief overview of our two datasets HydroSwot and PEPSI and a statistical analysis will be given. After detailing the methods, we will highlight the most important variables and define classes.  Second, we will concentrate on neural networks setup and theory, especially Artificial Neural Networks and Long Short-Term Memory Networks. Third, we will show how neural networks permit to predict river discharge. We will end this paper with a discussion about the different ways to improve the models we computed.

