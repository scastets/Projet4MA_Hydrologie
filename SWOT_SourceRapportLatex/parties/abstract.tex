\section*{Abstract.}

Managing water consumption and predicting flood risk are crucial issues which needs to better understand river flow on a global scale. Currently, data on rivers is poorly distributed geographically and temporally. Thus, in 2022, NASA and CNES will launch a satellite to collect altimetric data on rivers to reinforce the databases. However, a lack of knowledge exists in physical models which predict rivers flow, there is a bias on its value. We aim to use neural networks to reduce the bias and thus to obtain more accurate predictions. We performed an advanced statistical analysis on data to highlight the main parameters related to river discharge, and we built Artificial Neural Networks and Long Short-Term Memory Networks to estimate river discharge. We obtained satisfying results when the amount of data was sufficient, as the bias disappeared, but the predictions were not as high as expected when the dataset size was small. We conclude that our two different neural networks have two different aims, as an ANN can predict the river discharge for any river and the LSTM just for a particular river, and that the amount of data is the crucial point of the issue of the river discharge estimation.


